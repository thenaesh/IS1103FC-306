\section{Possible Mitigations}
 \subsection{Role and Nature of our Proposed Mitigations}
  \paragraph{}
        In view of the detrimental effects of FOMO, it is crucial for regulators to implement new rules in protection of technology users. Hence, this paper proposes three strategies that can be employed by the relevant authorities to mitigate the current state of matter with FOMO. Although these strategies might alter particular components of their targeted technologies, they are designed with the intention to preserve the nature and function of the service or system. It is only when these technologies are rid of their undesirable traits, that the benefit of employing them will truly outweigh the costs incurred, allowing users, and with extension society at large to fully benefit from them.
 \subsection{Augmenting the "Like" System}
  \paragraph{}        
        The first proposed strategy is to augment the "like" rating system inherent in various SNS, such that the amount of positive votes received for uploaded content is not shown publicly. The "like" system in this case, refers to the mandatory popularity counter that is attached to all uploaded content on popular SNS such as Facebook and Instagram. Although the rating system acts as a convenient feature for users to provide instant and positive feedback towards the content they view, its prevalence has also lead to obsessive behaviour in users of these sites. Account holders become fixated with the notion of virtual reputation \cite{breslin2009social} - determined by the number of positive ratings they receive , which becomes their main source of the basic need of feeling competent. Moreover, users also get caught up with the notion of not missing out on virtually popular content, which they also become reliant on to fulfill the need for relatedness. Hence, SNS users find themselves experiencing FOMO once they are unable to satisfy any of the two obsessions mentioned, and in turn have their daily lives negatively impacted. 
   \paragraph{}
        Not showing the amount of likes garnered by a particular user publicly achieves a win-win situation for everyone. Users who take on the role of content producers will feel the same amount of gratification as before and thereby not lose their motivation to continue generating or sharing value-adding content. Also, users on the other end of the spectrum assuming the roles of viewers will be able to show their appreciation for content that interests them, without feeling insecure about how they will fare if they were to start posting new content. SNS users will then become less reliant on the service to ensure that their basic psychological needs are met, reducing the occurrence of technology-induced FOMO. On the whole, all relevant SNS that undergo this augmentation will enjoy having a more positive environment for its users, bringing the focus of these sites back to their initial motivations of facilitating sharing and social interaction. This way, SNS users will finally be able to participate and reap the benefits of SNS without worrying about succumbing to FOMO.
 \subsection{Removing the Leaderboard in Games}
  \paragraph{}
        Another strategy this paper proposes is the removal of leaderboards in games, such that gamers are motivated to play them solely for their content, and not because of FOMO. The establishment of any form of signposting in games that remind gamers of others' achievements is unhealthy, as they deprive gamers of their basic need of feeling competent. This causes gamers to experience FOMO, as they participate actively in video games solely because they do not wish to miss out on any of the rewarding experiences that the top gamers are shown to be enjoying. With the possibility that end-game conditions and game content are constantly refreshed by game designers to keep players hooked, newer gamers will perpetually be unable to fully match up to the top dogs, causing them to be obsessed with an activity that is fuelled by their sense of FOMO. Hence, it is justified that the presence of leaderboards or any form of signposting highlighting player achievement is undesirable in game design as it exploits feelings of FOMO in gamers.
  \paragraph{}
        The elimination of these forms of signposting is beneficial for gamers of all ranking, achieving a net benefit for everyone. Although gamers standing on top of published leaderboards might feel less glorified about their achievements, they are also spared from the clutches of their own feelings of FOMO. They no longer have to sink in hours on days to maintain their reign, and are finally able to participate in games less obsessively, regaining their second psychological need of autonomy. Gamers out of the leaderboard will also regain their sense of competence, as they are no longer reminded of how other players are faring better in the game. This way, all players will be motivated to game only because they truly enjoy the intrinsic content within video games, and will no longer suffer from the negative consequences of gaming addiction due to FOMO.
 \subsection{Educating Users about FOMO}
  \paragraph{}
        The final strategy is for regulators to educate technology users of the existence of FOMO as a TRPD, as well as the measures they can take to prevent themselves from suffering FOMO. Albeit idealistic, this strategy ensures that users will be well equipped with knowledge on how to defend themselves against all forms of FOMO, even when they are faced with situations that are not mentioned in this paper. As compared to the two other strategies mentioned earlier, education is a long-term approach that might not take effect immediately, but will prove to have a larger impact than any other approaches to the problem of FOMO. Awareness is key and will be an effective solution to the problem of FOMO as a TRPD.
  \paragraph{}
        Propagation of information about FOMO can be expedited by the very channels that are causing it today, namely SNS and gaming portals themselves. As long as regulators are able to convince stakeholders of these channels to aid them in their cause, the very users who are most susceptible to FOMO as a TRPD will be receiving valuable information and precaution directly. Once a big enough stakeholder agrees and believes in the negative impacts of FOMO, the others will follow suit, either because they become convinced as well, or for the sake of fulfilling their Corporate Social Responsibility. Either way, technology users of the present and future will benefit from the spread of awareness of FOMO, and will finally be able take full advantage of new technologies without worrying about potential backlashes.