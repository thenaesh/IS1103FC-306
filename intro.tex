\section{Introduction}
  \subsection{Motivation}
    \paragraph{}
      Psychological disorders have existed throughout medical history. However, in recent years, the advent of cybertechnology has caused the flourishing of certain kinds of behaviour that can be characterised as psychological disorders. These new kinds of behaviour have their roots in the fundamental human condition that long predates cybertechnology, and yet have only come to the fore in the cybertechnology era. This necessitates the question of whether cybertechnology has been instrumental in promoting these kinds of behaviour.
    \paragraph{}
      This paper will explore the effects of cybertechnology on psychological disorders. We will refer to these disorders as technology-related psychological disorders (TRPD). TRPDs are worth looking into because of the adverse effects that they can have on those suffering from them. In this paper, we will give a major and representative example of a TRPD - namely fear of missing out (FOMO) - and its consequences. We will also discuss strategies for mitigating its effects.
  \subsection{Definitions}
    \paragraph{}
      A psychological disorder is a mental or behavioral pattern or anomaly that causes either suffering or an impaired ability to function in ordinary life. Such a psychological disorder becomes a TRPD if it is exacerbated by technology to the extent that is dwarfs any instances of that disorder that predate cybertechnology.