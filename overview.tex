\section{Overview of FOMO}
  \subsection{Definition}
    \paragraph{}
      Fear of missing out (FOMO) is defined as a pervasive apprehension that others may be having rewarding experiences from which one is absent \cite{przybylski2013motivational}.
    \paragraph{}
      People suffering from FOMO often exhibit the characteristic of wanting to stay updated on what others are doing \cite{przybylski2013motivational}. This busybody tendency is a consequence of their wanting to not miss out on any rewarding experiences that others may be having.
  \subsection{Causes of FOMO}
    \paragraph{}
      The root cause of FOMO can be attributed to the “deficits in psychological needs satisfactions”. These can be reduced to 3 basic needs, which are competence (the ability of one to have an impact on the world), autonomy (one’s self- authorship and initiative) and relatedness (how connected one is with others) \cite{przybylski2013motivational}.
    \paragraph{}
      Social media allows greater exposure to activities one might otherwise not know about, increasing the number of opportunities for one to participate in activities. However, an individual's capacity to participate in these activities remains the same, due to time and resource limitations. As a result, individuals can now participate in only a small fraction of the activities that they are exposed to \cite{przybylski2013motivational}. The regret and anxiety that might follow from missing out on these activities can then induce feelings of FOMO. A survey conducted by Vaughn et al also shows that 77\% of adult millennials (aged 18 -34) “often think that they can squeeze more than they really possibly can in (their) day”, and 58\% of them think that they “often spread (themselves) too thin, for fear of missing out” \cite{Vaughn2012Fear}, showing that their ability to participate in activities is indeed limited compared to the opportunities they can access. In the same survey, it is shown that 54\% of this group would feel left out when they “see that some of (their) friends or peers are doing something and (they’re) not”. Given that most of the posts and related activities in social media one would be exposed to would be related to their friends or peers, and the feeling of being left out would compromise the psychological need of relatedness, thus inducing FOMO due to that deficit in satisfaction of a psychological need, this shows that not being involved in the activities one is exposed to in social media can indeed result in increased feeling of FOMO.
    \paragraph{}
      There also seems to be a need for people to stay informed, which is also related to the basic psychological needs of relatedness, as being informed would help one feel more connected to others who are also informed of the same events or who are involved in those events. Referencing the survey by Vaughn et al, 83\% of the survey respondents said they like to be in the know. While social media increases the access to information which may seem to make the task of staying informed easier, this is a double edged sword as the increase in volume of available information also creates the expectation of having to know more information in order to be considered informed, and the inability to meet this expectation to remain in the know can also result in a feeling of FOMO. In the survey by Vaughn et al, 55\% of those aged 18 – 34 said that they are “overwhelmed by the amount of information (they) need to digest to stay up to speed”, showing that there is, at least to a slight majority of them, a great amount of information required in order to be considered informed, and that this amount might be exceeding, or at least stretching, the limit that they can manage, such that it may be considered “overwhelming”.
    \paragraph{}
      In addition, the exposure to positive details of the life of others in social media, through positive posts or pictures, which could display their fulfillment of the basic psychological needs, may result in one’s life appearing to be bland and unfulfilled in comparison, possibly resulting in a diminished sense of satisfaction of one’s own psychological needs of competence, autonomy and relatedness even if the satisfaction on an absolute level is not any worse off, therefore increasing the sense of FOMO due to a perceived deficit in these needs. As stated in research by Chou and Edge “looking at happy pictures of others on Facebook gives people an impression that others are ‘‘always’’ happy and having good lives, as evident from these pictures of happy moments” and “In contrast to their own experiences of life events, which are not always positive, people are very likely to conclude that others have better lives than themselves” \cite{chou2012they}. This relative deprivation, which arises from comparing their lives with others and thinking that they are worse off, intensifies the feeling of missing out \cite{Vaughn2012Fear}.
  \subsection{Why FOMO is a TRPD}
    \paragraph{}
      Even before technology, such needs were relevant to individuals, who sought ways to fulfill them through social events. FOMO was induced among individuals who were unable to attend such social events. However, with the influence of technology, namely social media such as Facebook, the feelings of FOMO have been intensified through various ways. As the survey results from a research by Vaughn et al in 2013, 44 – 48\% of those aged 13-47 agree that any FOMO that they have has been amplified by social media \cite{Vaughn2012Fear}.
    \paragraph{}
      The fact that technology has exacerbated FOMO significantly shows that FOMO is a TRPD.
      