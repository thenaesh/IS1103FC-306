\section{Examples of FOMO}
  \subsection{Social Networking and Instant Messaging Addiction}
    \paragraph{}
      The fear of missing out also manifests itself as another possible psychological disorder – Social Networking Sites (SNS) addiction. SNS addiction here refers to excessive habitual or compulsive use of or need for social networking sites, which can affect the SNS addict’s life in various ways, from inducing feelings of unhappiness to affecting their everyday life.
    \paragraph{}
      Social networking sites present their users with copious amounts of information regarding possible social activities, opportunities, or simply bits of interesting knowledge, which one suffering from FOMO may not want, or even find it unacceptable, to miss out on, which may lead to a compulsive need to use SNS. As Tom Lowery said, referencing a survey done by MyLife, a social media management platform, 56\% of consumers “believe not regularly checking their sites means they’ll miss important updates, news content or events from the pages they follow” \cite{lowery2013why}. Missing out on these updates and events, which one considers important, would be perceived to be a problem for one with the fear of missing out, hence FOMO can drive one to regularly check one’s social networking sites so as to avoid this problem of missing out. This drive may escalate to such a point that it may be considered as addiction to SNS. In an article by Samantha Murphy Kelly, for 27\% of survey respondents, the first thing they do after they wake up is to check their SNS, and 26\% of people would even trade habits like smoking or reality TV for access to SNS, which suggests the high priority these people consider SNS to be, and the dependence that they have on SNS \cite{kelly2013report} , which may border on addiction.
    \paragraph{}
      The overuse of SNS can negatively affect a person’s ability to function in everyday life, and bring about detrimental effects his life. Heavy use of SNS may mean that there is less time for other activities in real life, and possible neglect towards other aspects of one’s life from investing too much attention and effort on SNS. An example would be the impact of Facebook on one’s academic studies. In research by Kirschner and Karpinksi on university students, 74\% of those reporting on the impact of Facebook on their lives have said that Facebook indeed had negative impacts on their lives, specifically “procrastination, distraction, and poor time-management”, which would likely compromise one’s academic pursuits, as shown in the same research, where heavy Facebook use was seen among those who had lower Grade Point Averages \cite{kirschner2010facebook}. Being able to take timely action, focus and manage time can be considered as abilities that enable one to function in everyday life, thus the decrease in such abilities through increasing procrastination, distraction and poor time management due to Facebook use can be said to harm a person’s ability to function in ordinary life, and cause negative consequences like lowered academic performance. The compulsive need to use SNS that arises from FOMO can drive one to check their electronic devices even while operating motor vehicles \cite{przybylski2013motivational}, which can be dangerous due to the distraction that results from using SNS which may result in a higher chance for accidents to occur. In a study by the Transport Research Laboratory, use of a smartphone for social networking resulted in a 37.6\% increase in reaction time in a driving simulation, which is even higher than for those under the effects of cannabis, which is 21\%, and they also often missed important events on the road completely \cite{trl2012dont}. This shows that the use of SNS while driving indeed impairs driving ability, and thus the compulsion to use SNS even while driving due to FOMO has the potential to bring physical harm to an individual.
    \paragraph{}
      As previously stated, the exposure to positive details of the life of others in social networking sites through positive posts or pictures may result in one`s life appearing to be bland and unfulfilled in comparison, leading to possible feelings of envy or sadness, or in extreme cases, depression. This not only leads to dissatisfaction arising from the perception that one is worse off than others, but also envy towards those that one perceives to be better off. According to research by Tandoc et al, “heavy Facebook users have higher levels of Facebook envy than light users”, showing a positive correlation between Facebook use and Facebook Envy, and in the same research, it is shown that frequent feelings of envy might lead one to develop depression symptoms over time \cite{tandoc2015facebook}. This would be especially prominent for Facebook addicts, as considering the high Facebook use of one that is addicted to Facebook, and the positive correlation between Facebook use and Facebook envy, those addicted to Facebook are likely among the ones who often feel envious, which may then result in them developing symptoms of depression. Hence, frequent use of SNS, and especially addiction to SNS, can have various harmful effects on a person’s emotional well-being.
  
  
  \subsection{Online Gaming Addiction}
    \paragraph{}
      Another medium through which technology-induced FOMO can be observed would be online games.
    \paragraph{}
      Online gaming allows players to interact with each other, as well as with the game world. In online games, there is often an objective for players to achieve, as well as many small incentives which provide appeal to players. A fear of missing out could arise when players become strongly compelled to achieve these incentives or goals, and this could potentially lead to obsessive and extreme behaviour.
    \paragraph{}
      Part of the allure of online gaming lies in the ability of players to communicate and collaborate with people all over the world. Besides merely being able to interact with other players, they are also able to form groups, with which they overcome challenges to obtain achievements and other in-game perks while bonds are fostered amongst the team members \cite{johnson2015all}. They are also able to create organisations and factions within the game based on shared values, such as beliefs, goals or preferences. This allows players to interact with other like-minded people, with whom they feel a connection \cite{lo2005physical}. Other appeals of online gaming include satisfaction from completing difficult tasks, as well as the autonomy that comes with virtual reality. The feeling of satisfaction obtained from completing difficult tasks can be attributed to two main causes: one, the in-game perks that come with overcoming the challenge, and two, the sense of mastery that the gamer receives from overcoming the challenge \cite{johnson2015all}.
    \paragraph{}
      As mentioned previously, the Self-Determination Theory by Deci and Ryan speculates that a healthy psyche can be achieved if three basic psychological needs are fulfilled; Competence – the ability to effectively make a difference, Autonomy – the freedom and ability to control one’s own actions, and Relatedness – a feeling of connection with others \cite{przybylski2013motivational}. From the earlier discussion, it can be seen that online gaming is able to fulfil all three basic psychological needs. By overcoming challenges and unlocking in-game perks, the player is able to achieve confidence in his/her abilities. The virtual space grants players autonomy to explore and interact with the game world in ways that they would not be able to in real life, in roles that they may not be able to have in real life. In online games, a player can formulate exactly what kind of character he/she plays in the world, while in real life such a choice is not possible. Finally, interaction with like-minded individuals playing the same game, while fighting to achieve the same goals, give players a sense of connection. As such, it is a logical to conclude that online gaming presents an opportunity for players to achieve satisfying experiences, which they actively seek out. However, the design of online games is such that there are always additional objectives for players to obtain. This is something that online game designers do mindfully, as it prolongs interest in the online game and generates profit for the company. There are two main ways in which this can be done. The first is to divide players into different segments and to create targeted content for each group. The second is to increase the perceived value of in-game items to players. \cite{hamari2010game} The allure that online games hold over players, as well as the great amount of in-game opportunities designed into online games mean that players are susceptible to FOMO. In order to pursue the satisfying experiences that online gaming has the power to supply, players may be driven to try and achieve more in-game opportunities, fearing that they are missing out on experiences that are offered by the game. If not properly managed, such an inclination could easily become an obsession for online gaming.  
    \paragraph{}
      Obsessive behaviour towards online gaming could be classified as a sort of addiction, the impacts of which can seriously hinder the player’s psychological well-being, as well as his/her ability to function normally in everyday life. These impacts include feeling different from others, preoccupation with online gaming, and loss of control, amongst other consequences. The feeling of being different causes sufferers to experience feelings of isolation and restlessness, while also depriving the sufferer of one of the three basic psychological needs – relatedness. Preoccupation with online gaming means that the sufferer’s daily life is adversely affected, as he/she dedicates excessive amounts of time normally allocated to other activities, such as working or studying, to online gaming. Loss of control causes the sufferer to be unable to control his/her actions, often having uncontrollable bouts of obsessive behaviour, as well as a selective loss of memory when it comes to the negative impacts of their addiction. It can be seen that online game addiction deprives the sufferer of yet another of the three basic psychological needs – autonomy. \cite{sussman2011considering}
    \paragraph{}
      Online gaming addiction, which may exist for various reasons, is no doubt exacerbated by FOMO. As discussed above, online games provide a medium for players to obtain rewarding emotions, often from achieving certain objectives. Online game designers make use of this by creating an almost endless amount of opportunities, which players feel compelled to pursue, as they fear missing out on satisfying experiences. This creates a risk of online game addiction, especially if the player does not handle such inclinations adequately. As the onset of online game addiction causes suffering to the player by disrupting his/her psychological well-being, while also adversely affecting his/her ability to function in everyday life due to his/her obsessive behaviour, online game addiction can, and should, be considered a technology-related psychological disorder.